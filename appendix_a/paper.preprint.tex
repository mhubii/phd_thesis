\chapter[LBR-Stack]{LBR-Stack: ROS 2 and Python Integration of KUKA \gls{fri} for Med and
IIWA Robots}
\label{app:lbr_stack}
\minitoc

\paragraph{Disclaimer} This \appref{app:lbr_stack} is an \textit{in extenso} reproduction of~\cite{huber2023lbr}.

\newpage

\begin{figure}
\centering
\includegraphics[width=\textwidth]{appendix_a/img/joss_figure.png}
\caption[Supported robots in the \gls{lbr}-Stack. From left to right: KUKA \gls{lbr}
IIWA7, IIWA14, Med7, Med14. Visualizations made using Foxglove
.]{Supported robots in the \gls{lbr}-Stack. From left to right: KUKA \gls{lbr}
IIWA7, IIWA14, Med7, Med14. Visualizations made using Foxglove
\footnotemark{}.}
\end{figure}
\footnotetext{Foxglove: \url{https://foxglove.dev/ros}.}

\hypertarget{summary}{%
\section{Summary}\label{summary}}
The \texttt{\gls{lbr}-Stack} is a collection of packages that simplify the
usage and extend the capabilities of KUKA's Fast Robot Interface (\gls{fri})
\cite{ref-fri}. It is designed
for mission critical hard real-time applications. Supported are the
\texttt{KUKA\ \gls{lbr}\ Med7/14} and \texttt{KUKA\ \gls{lbr}\ IIWA7/14} robots in
the Gazebo simulation~\cite{ref-gazebo} and for communication with real hardware. A demo video can be
found
\href{https://www.linkedin.com/posts/mhubii_robotics-opensource-ros2-activity-7009974676017848320-S3U5/?utm_source=share\&utm_medium=member_desktop}{here}.
An overview of the software architecture is shown in Figure
\ref{fig:fri}.

At the \texttt{\gls{lbr}-Stack}'s core are two packages:

\begin{itemize}
\item
  \textbf{fri}: Integration of KUKA's original \gls{fri} client library into
  CMake.
\item
  \textbf{fri\_vendor}: Vendor library that integrates the \textbf{fri}
  into the \gls{ros} 2 build sytem.
\end{itemize}

All other packages are built on top. These include Python bindings and
packages for integration into the \gls{ros} and \gls{ros}
2:

\begin{itemize}
\item
  \textbf{pyFRI}: Python bindings for the \textbf{fri}.
\item
  \textbf{lbr\_fri\_ros2\_stack}: \gls{ros} 1/2 integration of the
  \texttt{KUKA\ \gls{lbr}}s through the \textbf{fri\_vendor}.
\end{itemize}

For brevity, and due to the architectural advantages over \gls{ros}
\cite{ref-ros2}, only \gls{ros} 2 is
considered in the following. The \textbf{lbr\_fri\_ros2\_stack}
comprises the following packages:

\begin{itemize}
\item
  \textbf{lbr\_bringup}: Python library for launching the different
  components.
\item
  \textbf{lbr\_description}: Description files for the \texttt{Med7/14}
  and \texttt{IIWA7/14} robots.
\item
  \textbf{lbr\_demos}: Demonstrations for simulation and the real
  robots.
\item
  \textbf{lbr\_fri\_msgs}: Interface Definition Language (IDL)
  equivalent of \gls{fri} protocol buffers.
\item
  \textbf{lbr\_fri\_ros2}: \gls{fri} \gls{ros} 2 interface through
  \texttt{realtime\_tools}~\cite{ref-ros_control}.
\item
  \textbf{lbr\_ros2\_control}: Interface and controllers for
  \texttt{ros2\_control}~\cite{ref-ros2_control}.
\item
  \textbf{lbr\_moveit\_config}: MoveIt 2 configurations
 ~\cite{ref-moveit}.
\end{itemize}

\begin{figure}
\centering
\includegraphics[width=\textwidth]{appendix_a/img/fri_dependency_architecture.pdf}
\caption[An overview of the overall software architecture. There exists
a single source for KUKA's \gls{fri}. This design facilitates that downstream
packages, i.e.~the Python bindings and the \gls{ros} 2 package, can easily
support multiple \gls{fri} versions. The \gls{ros} 2 side utilizes
vcstool.\label{fig:fri}]{An overview of the overall software
architecture. There exists a single source for KUKA's \gls{fri}. This design
facilitates that downstream packages, i.e.~the Python bindings and the
\gls{ros} 2 package, can easily support multiple \gls{fri} versions. The \gls{ros} 2 side
utilizes vcstool\footnotemark{}.\label{fig:fri}}
\end{figure}
\footnotetext{vcstool: \url{https://github.com/dirk-thomas/vcstool}.}

\hypertarget{statement-of-need}{%
\section{Statement of need}\label{statement-of-need}}

An overview of existing work that interfaces the KUKA LBRs from an
external computer is given in Table 1. We broadly classify these works
into custom communication solutions
\cite{ref-iiwa_stack,ref-kuka_sunrise_toolbox,ref-libiiwa} and
communication solutions through KUKA's \gls{fri} UDP channel 
\cite{ref-iiwa_ros2,ref-iiwa_ros}. The
former can offer greater flexibility while the latter offer a well
defined interface and direct software support from KUKA. Contrary to the
custom communication solutions, the \gls{fri} solutions additionally enable
hard real-time communication, that is beneficial for mission critical
development. Stemming from translational medical research, this work
therefore focuses on the \gls{fri}.

Limitations with the current \gls{fri} solutions are:

\begin{enumerate}
\def\labelenumi{\arabic{enumi}.}
\item
  Only support \texttt{IIWA7/14} robots, not \texttt{Med7/14}.
\item
  Do not provide Python bindings.
\item
  Maintainability:

  \begin{itemize}
  \item
    Modified client source code
    \href{https://github.com/epfl-lasa/iiwa_ros}{iiwa\_ros}.
  \item
    \gls{fri} client library tangled into source code
    \href{https://github.com/ICube-Robotics/iiwa_ros2}{iiwa\_ros2}.
  \end{itemize}
\item
  Partial support of \gls{fri} functionality. Both,
  \href{https://github.com/epfl-lasa/iiwa_ros}{iiwa\_ros} and
  \href{https://github.com/ICube-Robotics/iiwa_ros2}{iiwa\_ros2},
  exclusively aim at providing implementations of the \gls{ros} 1/2 hardware
  abstraction layer. This does not support:

  \begin{itemize}
  \item
    \gls{fri}'s cartesian impedance control mode.
  \item
    \gls{fri}'s cartesian control mode (\gls{fri} version 2 and above).
  \end{itemize}
\end{enumerate}

The first original contribution of this work is to add support for the
\texttt{KUKA\ \gls{lbr}\ Med7/14} robots, which, to the best author's
knowledge, does not exist in any other work. The second novel
contribution of this work is to provide Python bindings. This work
solves the maintainability by outsourcing the \gls{fri} into the separate
\textbf{fri} and \textbf{fri\_vendor} packages, which leaves the \gls{fri}'s
source code untouched and simply provides build support. 4. is solved by
defining an IDL message to KUKA's \texttt{nanopb} command and state
protocol buffers in \textbf{lbr\_fri\_msgs}. These messages can then be
interfaced from \gls{ros} 1/2 topics or from the \gls{ros} 1/2 hardware abstraction
layer.

\begin{landscape}
\begin{table}
\centering
\caption{Overview of existing frameworks for interfacing the KUKA LBRs.
A bullet point indicates support for the respective feature.}
\centering
\resizebox{0.7\paperheight}{!}{
\begin{tabular}{llllllllllll}
\hline
Framework                                                                      & IIWA      & Med       & \gls{ros}       & \gls{ros} 2     & Hard Real-time & \gls{fri}       & pyFRI     & Position Control & Impedance Control & Cartesian Impedance Control & Hardware Interface \\ \hline
$\href{https://github.com/lbr-stack}{\text{lbr-stack}}$                        & $\bullet$ & $\bullet$ & $\bullet$ & $\bullet$ & $\bullet$      & $\bullet$ & $\bullet$ & $\bullet$        & $\bullet$         & $\bullet$                   & $\bullet$          \\
$\href{https://github.com/epfl-lasa/iiwa_ros}{\text{iiwa\_ros}}$               & $\bullet$ &           & $\bullet$ &           & $\bullet$      & $\bullet$ &           & $\bullet$        & $\bullet$         &                             & $\bullet$          \\
$\href{https://github.com/ICube-Robotics/iiwa_ros2}{\text{iiwa\_ros2}}$        & $\bullet$ &           &           & $\bullet$ & $\bullet$      & $\bullet$ &           & $\bullet$        & $\bullet$         &                             & $\bullet$          \\
$\href{https://github.com/IFL-CAMP/iiwa_stack}{\text{iiwa-stack}}$             & $\bullet$ &           & $\bullet$ &           &                &           &           & $\bullet$        & $\bullet$         & $\bullet$                   &                    \\
$\href{https://github.com/Toni-SM/libiiwa}{\text{libiiwa}}$                    & $\bullet$ &           & $\bullet$ & $\bullet$ &                &           &           & $\bullet$        & $\bullet$         & $\bullet$                   &                    \\
$\href{https://github.com/Modi1987/KST-Kuka-Sunrise-Toolbox}{\text{KST-KUKA}}$ & $\bullet$ &           &           &           &                &           &           & $\bullet$        & $\bullet$         & $\bullet$                   &                    \\ \hline
\end{tabular}
}
\end{table}
\end{landscape}

\hypertarget{acknowledgement}{%
\section{Acknowledgement}\label{acknowledgement}}

We want to acknowledge the work in
\cite{ref-iiwa_stack}, as
their MoveIt configurations were utilized in a first iteration of this
project.
