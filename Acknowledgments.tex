First and foremost, I would like to express my dearest gratitude to both my supervisors Christos and Tom. Thank you for giving me the chance to come work with you at King's College in London, in this vibrant environment. It has been a once in a lifetime opportunity to come here. It has not always been the straightest route. We worked through COVID together, you kept the spirits up, despite your hardship of working extra hours and schooling your children. You were forgiving when things went the wrong way, and encouraged me to keep pushing. Thank you for being available day and night, during holidays and weekends, for proof-reading every paper, for taking the time to meet regularly, for hinting me in the right direction and listening to stupid ideas, thank you for creating opportunities, letting me go to places, and exposing me to all your network. I cannot thank you enough. It is only for your brilliance, dedication, and support that this thesis exists. I hope this is just the start and that we can continue on this journey together.

Next, I would like to thank the thesis progression commitee (TPC): Adelaide, Kawal, Hongbin, Prashant.  Your feedback helped shape the research targets immensely. I would like to express a special thanks to Adelaide, the TPC chair, for always being available and responding to requests promptly. 

Thank you S\'{e}bastien, Mark, and Jorge for creating these beautiful groups that lead to the creation of this laboratory, office, and the amazing integration into the hospital.

Furthermore, I would like to thank the RViM team for insightful discussions, positive feedback, help with the thesis, as well as experiments, and many good memories: John, Harris, Carlo, Anestis, Hans, Brandon, Zicong, Jeremy, Mirroyal, Jeref, Maleeha, Claudio, Gongyu, Soutiris, Zisos, Theo, and Hadi. A special thanks goes to Lyndon, for inviting me to many eye surgeries, for sharing your knowledge, for your growth mindset, and your pragmatism, many of which have been incredible learnings.

%KCL
Thank you Chayanin, Ross. Thank you for being great friends and making me feel at home in London. Thank you for all your technical recommendations, knowledgeable feedback, for thinking through ideas, setting things up, for maintaining a healthy working environment and for not taking things too seriously despite the exhausting working hours of a researcher that is. Only for your help, this submission turned into what it is today. I hope we can continue being friends and travel together in the future.

I would especially like to thank Christopher, Huanyu, Charlie, Dan, and Luis for extremely fruitful collaborations and plenty of technical feedback, much of which got incorporated into this work. It has been an absolute gift to get to work with you and I hope we get to work together again on another occasion in the future. I would love to learn a lot more from you and hear about your thoughts.

I would further like to thank Alejandro, Julien, Nicholas, and Yang for inviting me to a robotic surgery training, which provided some final insights to this thesis. I would especially like to thank Alejandro for being a great friend. I hope we get to go on many more runs together, maybe have Lucas help keep us accountable, whom I'd also like to thank.

I would like to thank the team in the optics lab and their help around the equipment, and just great input, and insights overall: Michael, Philip, Mirek, Theo, and Yijing. It has been incredible to get to bump into you on a daily basis, and I hope we will keep in touch. Thank you also, Yue, for motivating me to finish this thesis strong and for holding me accountable.

Thank you also, Samuel, Lilly, Mikel, and Viktor, my roommates throughout this journey. Thank you for being good roommates, for valuable discussions that helped inspire this research during many occasions.

Thank you to Theo, Maxence, Tiarna, Virginia, Pedro, Mark, Aaron, Remi, Tosin, Stephen, and Reuben. Thank you for being available when no one was around. Thank you for teaching how to submit jobs to the DGX, for your great inspiration, and for organizing Tommies' Social.

Thank you to the KCL staff, for keeping things running. I would especially like to thank the IT team for supporting the GPU cluster and providing the bouncer server: Andrew and Davide. Thank you also to Laurence for always helping with purchase orders swiftly. Thank you Irina for being a good friend and for keeping us in the pipe. I would further like to thank David for getting the laboratories up and running promptly during COVID, and for helpful discussions. My gratitude further goes to the SIE managers: Valentina, Marty, Duane, and Gayathri. Thank you for always helping with printing, moving, custom building screws and metal blocks, and for finding equipment. Without you, none of the experiments would have been possible. I would further like to thank the security personnel and the cleaning lady. Thank you for letting me into Becket House and the SIE laboratories day and night. Thank you further for keeping my desk in an orderly state.

%KCL staff
I am deeply thankful for meeting Konrad, for him giving me the opportunity to work in the US during an internship. Thank you for all your hard work, your mentorship, and for making me feel at home. Some of the learnings are now ingrained in this work, too. I would further like to thank his team for their continued help on-site, for playing volleyball together, going to hikes, and many more occasions: Eva (Yuqing), Olivier, Bryan, Ahmed, Alisha, Alan. Thank you also, Kathleen, for all your help.

%FAROS
I am very thankful for the collaborations with the entire FAROS team. Pulling a project of this scale off has not always been easy and required heavy lifting work from many participants. Thank you to the team from Leuven: Manu, Ayoob, Maikel, Ruixuan, and Kaat, for their input and support, especially for letting us use their robot. Thank you to the team from Zurich: Philipp, Fabio, Frederic, Nicola, and Aidana. A special thanks to Fabio, for having me stay on his couch during one of the integration weeks and for being a good friend. Thank you also to Frederic, for the joint work on robot drivers. Thank you to the team from Paris: Guillaume, Antoine, Jimmy, Francois, Saman, Lilian, Thibault, and Ellie. Many of the experiments that contributed to this thesis would not have been possible without your work.

%Familie
Nun auf meiner Muttersprache. Danke Thomas, Christian und Matthias dafür gute Brüder zu sein. Euer gutes Vorbild, TODO

Danke auch an Guan Teck und Philipp, dafür gute Freunde zu sein. Wär hätte gedacht, dass diese Arbeit mit einem Jobs Zitat startet, aber nun gibt es wohl kein Zurück mehr. Ein besonderer Dank an Guan Teck, lange ist es her, dass du mir dabei geholfen hast CT Binaries in C++ einzulesen, aber durch deine Hilfe bei diesem ersten Schritt wurde der Rest ermöglich.  

Weitherhin möchte ich meinen ehemaligen Mitstudenten und Freunden aus Heidelberg danken. Danke Lucas und Lucas, unser ständiger Ideenaustaucht, insbesondere durch unseren Journal Club während COVID, aber auch im Allgemeinen, hat einige Einsichten dieser Arbeit geprägt. Danke auch dafür, gute Freunde zu sein. Dein Besuch in London zu Beginn der Arbeit, Lucas, war ein wichtiges Erlebnis und er hat mir sehr dabei geholfen meinen Weg zu finden. Danke auch an Markus und dafür, dass ich bei dir zu Ende der Masterarbeit wohnen durfte, was das alles irgendwie in Gang gebracht hat. Danke auch an deine Oma.

Zu guter Letzt möchte ich dem Team der International Feedaz für ihren anhaltenden Hype danken, der die ein oder andere dunkle Nacht erleuchtet hat. Von Top bis Bot: Lukas, Hendrik und Viktor mit Jungle Diff Dennis.
