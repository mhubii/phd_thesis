\subsection{Point-to-plane \acrshort{icp}: A Lie algebra formulation}
\label{in:sec:p2plane}

\paragraph{Basic formulation.}
We first introduce the problem solved by the point-to-plane iterative closest point algorithm (point-to-plane \acrshort{icp})~\cite{Rusinkiewicz:IC3DIM:2001}.
Let $p_i$ and $q_i$ be corresponding 3D points in homogeneous coordinates. Let $n_i$ be a normal vector associated with $q_i$. We search a rigid body transformation $\Theta$ in $SE(3)$ that minimizes:
\begin{equation}
\sum_i ||n_i^T \cdot (\Theta \cdot p_i - q_i)||^2 = \sum_i (n_i^T \cdot \Theta \cdot p_i - n_i^T \cdot q_i)^2
\end{equation}

\paragraph{Iterative optimization on the $SE(3)$ Lie group.}
Let $\Theta_0$ be the current estimate. In a Lie group iterative optimization approach~\cite{Mahony:JGO:2002,Vercauteren:IPMI:2007}, we seek a perturbation $\delta\Theta \in \mathfrak{se}(3)$ composed with
$\Theta_0$: $\Theta_1 = \Theta_0 \cdot \exp(\delta\Theta)$. We now seek to minimize:
\begin{equation}
\sum_i (n_i^T \cdot \Theta_0 \cdot \exp(\delta\Theta) \cdot p_i - n_i^T \cdot q_i)^2
\end{equation}

\paragraph{Mathematical preliminaries.}
For convenience, we take advantage of the relationship between the 3D cross product, the Lie bracket on SO(3), and the skew-symmetric matrix operator.
Let $\omega = [\omega_x, \omega_y, \omega_z]$ and $\eta = [\eta_x, \eta_y, \eta_z]$ be two 3D vectors, we have
\begin{equation}
[\omega, \eta] = \omega \times \eta = \omega_\times \eta
\end{equation}
where
\begin{equation}
\omega_\times =
\begin{bmatrix}
0         & -\omega_z & \omega_y \\
\omega_z  & 0         & -\omega_x \\
-\omega_y & \omega_x  & 0
\end{bmatrix}
\end{equation}

It can be shown that the Lie group exponential in SO(3) admits a closed form through the Rodrigues’ formula. Let us consider $\omega$ as an element of $\mathfrak{so}(3)$, we now have
\begin{equation}
\expl(\omega) = \exp(\omega_\times) = \id + \frac{\sin(||\omega||)}{||\omega||}\omega_\times + \frac{1-\cos(||\omega||)}{||\omega||^2}\omega_\times^2
\end{equation}
which for small values of $||\omega||$ leads to the following numerically stable approximation
\begin{equation}
\expl(\omega) \approx \id
+ (1-\frac{||\omega||^2}{6}+\frac{||\omega||^4}{120})\omega_\times
+ (\frac{1}{2}-\frac{||\omega||^2}{24}+\frac{||\omega||^4}{720}))\omega_\times^2
\end{equation}

\iffalse
For compatibility with the 3D cross product, we choose the following basis for the matrix representation of $\mathfrak{se}(3)$:
\begin{align}
e_1 &= 
\begin{bmatrix}
0 & 0 & 0  & 0 \\
0 & 0 & -1 & 0 \\
0 & 1 & 0  & 0 \\
0 & 0 & 0  & 0
\end{bmatrix}
\\
e_2 &=
\begin{bmatrix}
0  & 0 & 1 & 0 \\
0  & 0 & 0 & 0 \\
-1 & 0 & 0 & 0 \\
0  & 0 & 0 & 0
\end{bmatrix}
\\
e_3 &=
\begin{bmatrix}
0 & -1 & 0 & 0 \\
1 & 0  & 0 & 0 \\
0 & 0  & 0 & 0 \\
0 & 0  & 0 & 0
\end{bmatrix}
\\
e_4 &=
\begin{bmatrix}
0 & 0 & 0 & 1 \\
0 & 0 & 0 & 0 \\
0 & 0 & 0 & 0 \\
0 & 0 & 0 & 0
\end{bmatrix}
\\
e_5 &=
\begin{bmatrix}
0 & 0 & 0 & 0 \\
0 & 0 & 0 & 1 \\
0 & 0 & 0 & 0 \\
0 & 0 & 0 & 0
\end{bmatrix}
\\
e_6 &=
\begin{bmatrix}
0 & 0 & 0 & 0 \\
0 & 0 & 0 & 0 \\
0 & 0 & 0 & 1 \\
0 & 0 & 0 & 0
\end{bmatrix}
\end{align}
\fi

Let us consider a 6D vector $u=[\omega;\tau]$ as an element of $\mathfrak{se}(3)$. Its matrix representation is provided by
\begin{equation}
u_\dagger =
\begin{bmatrix}
\omega_\times & \tau \\
0  & 0
\end{bmatrix}
\end{equation}
which leads to the following closed-form formula for the Lie group exponential for SE(3):
\begin{equation}
\expl(u) = \exp(u_\dagger) =
\begin{bmatrix}
\expl(\omega) & L(\omega)\tau \\
0  & 1
\end{bmatrix}
\end{equation}
where
\begin{equation}
L(\omega) = \id
+ \frac{1-\cos(||\omega||)}{||\omega||^2}\omega_\times
+ \frac{||\omega||-\sin(||\omega||)}{||\omega||^3}\omega_\times^2
\end{equation}
where a Taylor expansion should again be used for numerical stability if $||\omega||$ is small:
\begin{equation}
L(\omega) \approx \id
+ (\frac{1}{2}-\frac{||\omega||^2}{24}+\frac{||\omega||^4}{720})\omega_\times
+ (\frac{1}{6}-\frac{||\omega||^2}{120}+\frac{||\omega||^4}{5040})\omega_\times^2
\end{equation}

\paragraph{First-order linearization of the action of the exponential in $\mathfrak{se}(3)$.}
Let $p=[\tilde{p};1]$ be a 3D point in homogeneous coordinates. We consider an infinitesimal element $\delta u=[\delta \omega;\delta \tau]$ of $\mathfrak{se}(3)$ and consider the corresponding action on $p$:
\begin{equation}
\begin{split}
\expl(\delta u)p &\approx (\id+\delta u_\dagger)p = p +
\begin{bmatrix}
\delta \omega_\times & \delta \tau \\
0  & 0
\end{bmatrix} p
= p +
\begin{bmatrix}
\delta \omega_\times \tilde{p}+\delta \tau \\
0
\end{bmatrix} \\
&= p +
\begin{bmatrix}
\delta \omega \times \tilde{p}+\delta \tau \\
0
\end{bmatrix}
= p +
\begin{bmatrix}
-\tilde{p} \times \delta \omega + \delta \tau \\
0
\end{bmatrix} \\
&= p +
\begin{bmatrix}
-\tilde{p}_\times & \id \\
0 & 0
\end{bmatrix} \delta u
= p + D(p)\delta u
\end{split}
\end{equation}
where
\begin{equation}
D(p) = 
\begin{bmatrix}
-\tilde{p}_\times & \id \\
0 & 0
\end{bmatrix}
\end{equation}

\paragraph{Lie algebra linearization of the point-to-plane objective}
Plugging this in the original cost function we get the following linearization:
\begin{equation}
\begin{split}
\sum_i & \Big(n_i^T \Theta_0 \big(p_i + D(p_i)\delta\Theta \big) - n_i^T q_i\Big)^2
\\
&= \sum_i \Big(n_i^T \Theta_0 D(p_i)\delta\Theta + n_i^T \Theta_0 p_i - n_i^T q_i\Big)^2
\end{split}
\end{equation}
Denoting $a_i = n_i^T \Theta_0 D(p_i)$, $A=[a_0;\ldots;a_N]$,
$b_i = n_i^T (q_i - \Theta_0 p_i)$, and $b=[b_0;\ldots;b_N]$, we end up with a standard linear least squares problem:
\begin{equation}
||A \delta\Theta - B||^2
\end{equation}
which can be expressed using the pseudo-inverse $A^{\dagger}$ of $A$:
\begin{equation}
\delta\Theta = A^{\dagger} B
\end{equation}

Given the dimensions of $A$, $b$ and $\delta\Theta$ is probably best solved by using the normal equations
\begin{equation}
(A^T A) \delta\Theta = A^T B
\end{equation}
and relying on a Cholesky decomposition:
\begin{equation}
\delta\Theta = \texttt{chol\_solve}(A^T A, A^T B)
\end{equation}

Note that by expressing A and B without homogeneous coordinates, we get:
\begin{equation}
a_i = [-\tilde{n}_i^T R_0 {{\tilde{p}}_{i\times}}, \tilde{n}_i^T R_0]
\end{equation}
and
\begin{equation}
b_i = \tilde{n}_i^T (\tilde{q}_i - R_0 \tilde{p}_i - T_0)
\end{equation}

\paragraph{Robust formulation.}
To reduce the influence of outliers on the solution, a typical approach is to introduce a
loss function $\rho_\kappa$ parameterized by a scale parameter (a.k.a. soft margin or cut-off) $\kappa$ to scale the residuals. We then seek to minimize:
\iffalse
\begin{equation}
\sum_i \rho\big(\frac{1}{C^2}(n_i^T \cdot \Theta \cdot p_i - n_i^T \cdot q_i)^2\big)
\end{equation}

Using the identity loss function $\rho(z)=z$ leads to the original least-squares problem. Classical robust alternatives include:
\begin{description}
  \item[Soft L1]:
  $\rho(z) = 2 (\sqrt{1 + z} - 1)$.
  The smooth approximation of l1 (absolute value) loss. Usually a good choice for robust least squares.

  \item[Huber]:
  $\rho(z) = z \texttt{ if } z <= 1 \texttt{ else } 2\sqrt{z} - 1$.
  Works similarly to ``Soft L1".

  \item[Cauchy]:
  $\rho(z) = \ln(1 + z)$.
  Severely weakens outliers influence, but may cause difficulties in optimization process.

   \item[Arctan]:
   $\rho(z) = \arctan(z)$.
   Limits a maximum loss on a single residual, has properties similar to ``Cauchy".
\end{description}
\fi
\begin{equation}
\sum_i \rho_k\Big(n_i^T \cdot \Theta \cdot p_i - n_i^T \cdot q_i\Big)
\end{equation}
or equivalently if we start from a given estimate in the above Lie group setting:
\begin{equation}
\sum_i \rho_k\Big(a_i \delta\Theta - b_i\Big)
\end{equation}
%
Using the square loss function $\rho_\kappa(z)=z^2$ leads to the original least-squares problem. A classical robust alternatives is the Huber loss
\begin{equation}
\rho_\kappa(z) =
\begin{cases}
  z^2 &\texttt{ if } |z| <= \kappa \\
  2|z|\kappa - \kappa^2  &\texttt{ otherwise }
\end{cases}
\end{equation}
with $\kappa=1.345\sigma$ being advocated to provide robustness while retaining appropriate properties when the errors are Gaussian.

This minimisation can be achieved by a standard Iteratively Reweighted Least Squares (IRLS) approach~\cite{Green:JRSSB:1984} or by including a second order terms in a slightly more advanced reweighted Gauss-Newton approach as discussed in~\cite{Triggs:VisAlg:2000}.
The latter is referred to as the Triggs correction in \url{http://ceres-solver.org/nnls_modeling.html#theory}. While elegant, previous work has shown that the Triggs correction does not significantly improve on the standard IRLS~\cite{Zach:ECCV:2014,Zach:ECCV:2018}.
Here, for simplicity we thus restrict ourselves to standard IRLS.
Given a choice of scaling factor $\kappa$ and an estimate of the parameters $\Theta$, the robust problem is converted to a weighted least-squares:
\begin{equation}
\sum_i \omega_\kappa(b_i) ||a_i \delta\Theta - b_i||^2
\end{equation}
with $\omega_\kappa(z) = \rho'_\kappa(z) / z$ being the weight function associated with the Huber loss:
\begin{equation}
\omega_\kappa(z) =
\begin{cases}
  1 &\texttt{ if } |z| <= \kappa \\
  \kappa / |z|  &\texttt{ otherwise }
\end{cases}
\end{equation}

This leads us to the following weighted normal equation:
\begin{equation}
(A^T W A) \delta\Theta = A^T W B
\end{equation}
where $W = \diag\big(\omega_\kappa(b_0), \ldots, \omega_\kappa(b_N)\big)$.

One aspect we haven't addressed yet it the choice of $\kappa$ for the Huber loss. As mentioned earlier, a typical choice is to use $\kappa=1.345\sigma$. We are thus left with the need to robustly estimate the standard deviation of the residuals. This is typically done through a median absolute deviation (MAD):
\begin{equation}
\hat{\sigma} = \frac{\MAD(\{b_i\})}{0.6745} = \frac{\median(\{|b_i-\median(\{b_i\}|)\})}{0.6745}
\end{equation}

\subsection{Registration of a 3D point cloud with 2D stereo segmentations}
\label{in:sec:registration_of_3d_point_cloud}
%
Let $q_i$ be 3D points associated to a model object as expressed in an object specific coordinate frame.
Let $I_L$ and $I_R$ be a pair of stereo images with associated $M_L$ and $M_R$ camera matrices (including both extrinsic and intrinsic).
We assume that the stereo images are observing the object in an unknown pose $\Theta$.
We also assume that the object is relatively textureless and that we don't have a good means of extracting keypoints of interest from the images.
We however assume that a segmentation of the object in each image can readily be computed, e.g. with the Segment Anything Model (SAM)~\cite{Kirillov:arxiv:2023}.
%
While differentiable rendering could be used in this context~\cite{Hannemose:SPIE:2019,Labbe:CVPR:2021,Chen:RAL:2023},
we expect that a simpler algorithm can be designed given the segmentation availability by matching points from 3D to 2D.
We also expect to potentially gain robustness against illumination and other factors that would impact the appearance of the object in the images.

Let $S_L$ and $S_R$ be the object segmentation masks in the left and right image respectively.
We here seek to recover the pose $\Theta$ such that the projected model points ${M_L \Theta q}$ match the observed segmentation $S_L$ (respectively ${M_R \Theta q}$ to match $S_R$).
Let $\varphi$ be a function to transform from homogeneous to Cartesian coordinates.
That is $\varphi([x,y,f])=[x/f,y/f]$ and similarly in 3D.
We can write the objective as:
\begin{equation}
\arg\min_{\Theta} \sum_i
\min_{p_L|S_L(p_L)=1}|| \varphi\big(M_L \Theta q_i\big) - p_L||^2
+ \min_{p_R|S_R(p_R)=1}|| \varphi\big(M_R \Theta q_i\big) - p_R||^2
\end{equation}

Similar to the approach in~\cite{Fitzgibbon:IVC:2003}, this formulation can be substantially accelerated by taking advantage of the fact that the segmentations are defined on a regular grid. We can thus use a Euclidean Distance Transform (EDT) for which optimized algorithms exist on both CPU~\cite{Felzenszwalb:TC:2012}\footnote{\url{https://github.com/cai4cai/distance_transform}} and GPU~\cite{Cao:SIGGRAPH:2010}\footnote{\url{https://docs.rapids.ai/api/cucim/stable/api/\#cucim.core.operations.morphology.distance_transform_edt}}.
Let $\mathcal{D}_{S_L}$ be the EDT computed from $S_L$ (respectively $\mathcal{D}_{S_R}$ for $S_R$).
We can now write
\begin{equation}\label{eq:dt3d2d}
\arg\min_{\Theta} \sum_i \mathcal{D}_{S_L}^2\big( \varphi(M_L \Theta q_i) \big)
+ \mathcal{D}_{S_R}^2\big( \varphi(M_R \Theta q_i) \big)
\end{equation}

As in Section \ref{in:sec:p2plane}, the square loss terms could be replaced by robust alternatives.
It should further be noted that the evaluation of a distance transform $\mathcal{D}_{S}$ at an arbitrary point $p$ requires the use of an interpolator.
A typical choice is to rely on bilinear interpolation for which optimized implementations exist on both CPU and GPU\footnote{\url{https://pytorch.org/docs/stable/generated/torch.nn.functional.grid_sample.html}}.

Solving for \eqref{eq:dt3d2d} can be done again by taking advantage of the Lie group structure of SE(3). For simplicity, a generic Lie group non-linear least-squares optimizer as found in Theseus\footnote{\url{https://github.com/facebookresearch/theseus}}~\cite{Pineda:NeurIPS:2023} or Ceres\footnote{\url{https://github.com/ceres-solver/ceres-solver}}~\cite{Agarwal:Ceres:2022} can be used.

It is worth realising that even if a rigid body model is used, formulation \eqref{eq:dt3d2d} may have somewhat trivial solutions if configurations can be found where the projected 2D point set can fit completely inside of the object segmentation mask as this would lead to a zero loss. It may thus be advantageous to somewhat also consider points outside of the model and make sure that these stay outside of the segmentation mask.
