% instructions
% https://www.grad.ucl.ac.uk/essinfo/docs/Impact-Statement-Guidance-Notes-for-Research-Students-and-Supervisors.pdf

% no more than 500 words

% full instructions (260 words)
% The statement should describe, in no more than 500 words, how the expertise, knowledge, analysis, discovery or insight presented in your thesis could be put to a beneficial use. Consider benefits both inside and outside academia and the ways in which these benefits could be brought about. The benefits inside academia could be to the discipline and future scholarship, research methods or methodology, the curriculum; they might be within your research area and potentially within other research areas. The benefits outside academia could occur to commercial activity, social enterprise, professional practice, clinical use, public health, public policy design, public service delivery, laws, public discourse, culture, the quality of the environment or quality of life. The impact could occur locally, regionally, nationally or internationally, to individuals, communities or organisations and could be immediate or occur incrementally, in the context of a broader field of research, over many years, decades or longer. Impact could be brought about through disseminating outputs (either in scholarly journals or elsewhere such as specialist or mainstream media), education, public engagement, translational research, commercial and social enterprise activity, engaging with public policy makers and public service delivery practitioners, influencing ministers, collaborating with academics and non-academics etc. Further information including a searchable list of hundreds of examples of UCL impact outside of academia please see Research Impact website. For thousands more examples, please see REF2014 website. A workshop, How to Write an Effective Impact Statement, is run as part of the Doctoral Skills Development Programme to provide assistance to students in writing about the impact of their research.

The research outlined in this PhD thesis holds the promise of significant global impact across societal, clinical, and academic realms. While acknowledging its preliminary stage, this work has the potential to reshape clinical practices, redefine research paradigms, and elevate the stature of the university. As we cautiously assess its early impact, we anticipate its future contributions to be transformative and far-reaching

\paragraph{Societal Impact} Given its primary focus on automation, this thesis holds significant potential for societal impact, encompassing both advantageous and detrimental implications. On the positive spectrum, automation has the capacity to sustain healthcare accessibility within a society confronting labor shortages, potentially reducing associated costs. Moreover, it could facilitate the redirection of human efforts towards more enriching endeavors. Conversely, the advent of automation may precipitate job displacement or compel individuals into roles perceived as less fulfilling, constituting potential drawbacks of this technological shift.

\paragraph{Clinical Impact} This research has the potential to expedite clinical services by mitigating reliance on the presence of clinical personnel. As this thesis delves into the preliminary stages of camera motion automation, it offers a foundation for extrapolating insights into automating a broader spectrum of tasks in subsequent endeavors. Considering potential economic motivations, it is imperative to ensure that any newly devised systems remain within current cost ranges. It is essential to acknowledge that, akin to surgeons, automation carries the risk of patient harm in the event of errors. However, the methodologies developed within this thesis shed light on potential strategies to mitigate such risks even in automated settings.

\paragraph{Academic Impact} The academic significance of this research primarily lies in pioneering self-supervised learning methodologies for camera motion automation, ensuring safe implementation in clinical settings. These innovations have the potential to catalyze the emergence of new research domains while steering existing fields towards unexplored avenues. Notably, this study represents the inaugural exploration of automation prospects on a comprehensive scale using data from large-scale videos of laparoscopic surgeries. Although it remains premature to forecast additional advancements in automation stemming from this thesis, we remain optimistic about the prospect. Should further breakthroughs materialize, this research stands poised to enhance King's College's standing as a hub for pioneering research endeavors.
