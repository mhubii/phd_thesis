%%%%%%%%%%%%%%%%%%%%%%%%%%%%%%%%%%%%%%%%%%%%%%%%%
% Custom commands
%%%%%%%%%%%%%%%%%%%%%%%%%%%%%%%%%%%%%%%%%%%%%%%%%
\DeclareMathOperator{\IM}{\mathbf{DE9IM}}
\DeclareMathOperator{\Crop}{Crop}
\DeclareMathOperator{\Warp}{Warp}
\DeclareMathOperator{\diag}{diag}
\DeclareMathOperator{\expl}{exp_{Lie}}
\DeclareMathOperator{\id}{Id}
\DeclareMathOperator{\MAD}{MAD}

\newcommand{\homogeneous}[2]{{}^{^#1}\Theta_{_#2}}
\newcommand\algoref{Algorithm~\ref}

%%%%%%%%%%%%%%%%%%%%%%%%%%%%%%%%%%%%%%%%%%%%%%%%%
% Extra packages
%%%%%%%%%%%%%%%%%%%%%%%%%%%%%%%%%%%%%%%%%%%%%%%%%
\usepackage{caption}
\usepackage{subcaption}
\usepackage{booktabs}
\usepackage{lscape}
\usepackage{natbib}
\usepackage{adjustbox}
% \usepackage[ruled,vlined]{algorithm2e}
\usepackage{algorithm}
\usepackage{algpseudocode}
\usepackage{siunitx}
% \usepackage{hyperref}

%%%%%%%%%%%%%%%%%%%%%%%%%%%%%%%%%%%%%%%%%%%%%%%%%
% Custom settings
%%%%%%%%%%%%%%%%%%%%%%%%%%%%%%%%%%%%%%%%%%%%%%%%%
\setcounter{secnumdepth}{3}
\setcitestyle{square}

%%%%%%%%%%%%%%%%%%%%%%%%%%%%%%%%%%%%%%%%%%%%%%%%%
% Layout
%%%%%%%%%%%%%%%%%%%%%%%%%%%%%%%%%%%%%%%%%%%%%%%%%
\usepackage[left=1.5in,right=1.3in,top=1.1in,bottom=1.1in,includefoot,includehead,headheight=13.6pt]{geometry}
% Use 1.5 line spacing. UCL says
% "Double or one-and-a-half spacing should be used in typescripts,
% except for indented quotations or footnotes where single spacing may
% be used."
\usepackage{setspace}
\onehalfspacing

% Draft note
% No draft note (empty one) by default
\providecommand{\draftnote}{}

% Fancy Header and Footer
\usepackage{fancyhdr}
\pagestyle{fancy} % Sets fancy header and footer

\fancyhf{} % clear all header and footer fields

% line width used to draw the header line
\renewcommand{\headrulewidth}{1.0pt}

\makeatletter

% redefines chapter marks to use
% the chapter number only in mainmatter
\def\chaptermark#1{%
 \markboth {\MakeUppercase{%
 \ifnum \c@secnumdepth >\m@ne
  \if@mainmatter
   \@chapapp\ \thechapter. \ %
  \fi
 \fi
#1}}
{\ifnum \c@secnumdepth >\m@ne
  \if@mainmatter
   \@chapapp\ \thechapter. \ %
  \fi
 \fi
#1}}%

\if@twoside
 % Page number (boldface) in left on even
 % pages and right on odd pages
 \fancyhead[LE,RO]{\bfseries\thepage}
 % Chapter in the right on even pages
 \fancyhead[RE]{\bfseries\nouppercase{\leftmark}}
 % Section in the left on odd pages
 \fancyhead[LO]{\bfseries\nouppercase{\rightmark}}

 % Draft mark
 \fancyfoot[LE,RO]{\draftnote}

 % Page number and draft mark on plain pages
 \fancypagestyle{plain}{
   \fancyhf{} % clear all header and footer fields
   \fancyfoot[C]{\bfseries\thepage}
   \fancyfoot[LE,RO]{\draftnote}
   \renewcommand{\headrulewidth}{0pt}
 }

 % Draft mark on 'empty' pages
 \fancypagestyle{empty}{
   \fancyhf{} % clear all header and footer fields
   \fancyfoot[LE,RO]{\draftnote}
   \renewcommand{\headrulewidth}{0pt}
 }
\else
 \fancyhead[R]{\bfseries\thepage}
 \fancyhead[L]{\bfseries\nouppercase{\leftmark}}
 \fancyfoot[R]{\draftnote}
 \fancypagestyle{plain}{
   \fancyhf{} % clear all header and footer fields
   \fancyfoot[C]{\bfseries\thepage}
   \fancyfoot[R]{\draftnote}
   \renewcommand{\headrulewidth}{0pt}
 }
 \fancypagestyle{empty}{
   \fancyhf{} % clear all header and footer fields
   \fancyfoot[R]{\draftnote}
   \renewcommand{\headrulewidth}{0pt}
 }
\fi
\makeatother

% Clear Header Style on the Last Empty Odd pages
\makeatletter
\def\cleardoublepage{\clearpage\if@twoside \ifodd\c@page\else%
  \hbox{}%
  \thispagestyle{empty}%              % Empty header styles
  \newpage%
  \if@twocolumn\hbox{}\newpage\fi\fi\fi}
\makeatother



% Define a simple copyright page style
\newcommand{\copyrightpage}[2]{
\thispagestyle{empty}%              % Empty header styles
~\\
\vfill
\begin{center}
 {\copyright}#1\\
 #2\\
 All Rights Reserved
\end{center}
\clearpage
}

% Define a simple declaration page style
\newcommand{\declarationpage}[1]{
\thispagestyle{plain}%
I, #1, confirm that the work presented in this thesis is my own. Where
information has been derived from other sources, I confirm that this
has been indicated in the thesis.
\cleardoublepage
}


% Define a simple quotation page style
\newcommand{\quotationpage}[2]{
\thispagestyle{plain}%              % Empty header styles
\vspace*{\stretch{1}}
\begin{quotation}
\emph{#1}
\begin{flushright}
#2
\end{flushright}
\end{quotation}
\vspace*{\stretch{3}}
\clearpage
}

\usepackage{fancybox}
\setlength{\shadowsize}{1pt}

\usepackage{wallpaper}

\renewcommand{\contentsname}{Table of contents}
\renewcommand{\listfigurename}{List of figures}
\renewcommand{\listtablename}{List of tables}

% load monitoc with nohints to shut down
% unusefull warnings about hyperref
\usepackage[nohints]{minitoc}

% Set minitoc title to match the document contentsname
\mtcsettitle{minitoc}{\contentsname}

%%%%%%%%%%%%%%%%%%%%%%%%%%%%%%%%%%%%%%%%%%%%%%%%%
% Algorithms and theorems
%%%%%%%%%%%%%%%%%%%%%%%%%%%%%%%%%%%%%%%%%%%%%%%%%
\usepackage[amsmath,thmmarks,hyperref]{ntheorem}

\setlength{\theorempostskipamount}{15pt}

\theoremstyle{break}
\theoremheaderfont{\normalfont\bfseries}
\theoremseparator{}
\theorembodyfont{\normalfont}
\theoremsymbol{\rule{1ex}{1ex}}

%%%%%%%%%%%%%%%%%%%%%%%%%%%%%%%%%%%%%%%%%%%%%%%%%
% Fonts and Encoding
%%%%%%%%%%%%%%%%%%%%%%%%%%%%%%%%%%%%%%%%%%%%%%%%%
% Computer modern family
\usepackage{lmodern}
% If you need bold small caps with lmodern, uncomment
% the 2 following lines
%\rmfamily
%\DeclareFontShape{T1}{lmr}{bx}{sc}{<->ssub*cmr/bx/sc}{}

%%%%%%%%%%%%%%%%%%%%%%%%%%%%%%%%%%%%%%%%%%%%%%%%%
% Bibliography
%%%%%%%%%%%%%%%%%%%%%%%%%%%%%%%%%%%%%%%%%%%%%%%%%
\usepackage[style=alphabetic,maxnames=99,maxcitenames=1,giveninits=false,uniquename=init,backref=true]{biblatex}
% Make parenthical citations the default
\renewcommand\cite{\parencite}

\DeclareCiteCommand{\citet}
  {\boolfalse{citetracker}%
   \boolfalse{pagetracker}%
   \usebibmacro{prenote}}
  {\ifciteindex
     {\indexnames{labelname}}
     {}%
   \printtext[bibhyperref]{\printnames{labelname}~[\printfield{year}]}}
  {\multicitedelim}
  {\usebibmacro{postnote}}

\renewcommand*{\labelalphaothers}{}
\DeclareLabelalphaTemplate{
  \labelelement{
    \field[final]{shorthand}
    \field[names=1]{labelname}
    \field{label}
  }
  \labelelement{
    \literal{\addhighpenspace}
  }
  \labelelement{
    \field{year}
  }
}

%%%%%%%%%%%%%%%%%%%%%%%%%%%%%%%%%%%%%%%%%%%%%%%%%
% Hyperreferences
%%%%%%%%%%%%%%%%%%%%%%%%%%%%%%%%%%%%%%%%%%%%%%%%%
% choose link colors colors
\usepackage{color}
% For pdf
%\definecolor{linkcol}{rgb}{0,0,0.4}
%\definecolor{citecol}{rgb}{0.5,0,0}
%
% For printing
\definecolor{linkcol}{rgb}{0,0,0}
\definecolor{citecol}{rgb}{0,0,0}

\ifpdf
\usepackage[pdfusetitle,pdfpagelabels=true]{hyperref}
\else
\usepackage[dvips,pdfusetitle,pdfpagelabels=true]{hyperref}
\fi

\hypersetup
{
bookmarksdepth=3,
bookmarksopenlevel=0,
bookmarksnumbered=true,
draft=false, % force to use hyperref even in draft mode
%pdftoolbar=false, %barre d'outils non visible
%pdfmenubar=true, %barre de menu visible
plainpages=false,
pdfhighlight=/O, %effet d'un clic sur un lien hypertexte
colorlinks=true, %couleurs sur les liens hypertextes
pdfdisplaydoctitle=true,
pdflang=en-US, % Does not seem to work but does not harm
pdfpagemode=UseOutlines,
%pdfpagelayout=SinglePage, %ouverture en simple page
%pdffitwindow=true, %pages ouvertes entierement dans toute la fenetre
pdfstartview=FitH,
linkcolor=linkcol, %couleur des liens hypertextes internes
citecolor=citecol, %couleur des liens pour les citations
urlcolor=linkcol, %couleur des liens pour les url
pdfkeywords={keyword 1, keyword 2},
pdfsubject={PhD Thesis}
}
